\documentclass{article}
\usepackage{hyperref}
\usepackage{amssymb, mathtools, amsmath}
\usepackage[dvipsnames]{xcolor}
\usepackage{graphicx}
\usepackage{float}
\usepackage{caption}
\usepackage{hyperref}
\hypersetup{
    colorlinks=true,
    linkcolor=blue,
    filecolor=magenta,      
    urlcolor=blue,
}

\title{ Macroeconomic Theory
        \thanks{Course instructed by Professor Christoph Hedtrich.} \\
        Homework 6: Dynamic Stochastic General Equilibrium Models
        }

\author{
        % Add your name here
        Uppsala Masters in Economics 2021-2022
        }

\date{25 November 2021}

% margins
\oddsidemargin 3mm
\evensidemargin 3mm
\topmargin -12mm
\textheight 600pt
\textwidth 420pt

% no indent
\setlength\parindent{0pt}
% \renewcommand{\theenumi}{\thesection(\alph{enumi})}
% \renewcommand\thesubsection{\arabic{subsection}}

% custom commands
\newcommand{\E}[1]{\mathrm{E}\left[#1\right]}
\newcommand{\Et}[1]{\mathrm{E}_t\left[#1\right]}
\newcommand{\cov}[1]{\mathrm{Cov}\left(#1\right)}
\newcommand{\var}[1]{\mathrm{Var}\left(#1\right)}
\renewcommand{\L}{\mathcal{L}}
\newcommand{\?}{\textcolor{red}{(?)}} % question mark

\begin{document}
    
    \maketitle
    
    \section{Nominal Rigidity}
    
    \subsection{Assumptions}
        
        For simplicity, assume \textbf{firms} produce output using only labor,
        \begin{align}
            Y = F(L), \quad F'(\cdot) > 0, \quad F''(\cdot) \le 0.
        \end{align}
        
        \textbf{Representative household} maximizes lifetime utility
        \begin{align}
            \mathcal{U} &= \sum_{t=0}^\infty \beta^t
            \left[
                U(C_t) + \Gamma\left(\frac{M_t}{P_t}\right) - V(L_t)
            \right]
            \\
            s.t. & \quad \sum_{t=0}^\infty \frac{C_t}{\prod_{s=0}^{t}(1+r_s)}
            \le \sum_{t=0}^\infty \frac{W_t L_t}{\prod_{s=0}^{t}(1+r_s)},
        \end{align}
        
        where $C_t$ is consumption, $L_t$ is labor, where $M_t$ is the money holdings and $P_t$ are the fixed aggregate prices in the period. We call $\frac{M_t}{P_t}$ the \textbf{real money supply}. Real interest rate is defined as
        \begin{align}
            1 + r_t &= (1+i_t)\frac{P_t}{P_{t+1}}
            = \frac{1+i}{1+\pi_t}
        \end{align}
        where $i_t$ is nominal interest and $\pi_t$ is the inflation rate. The household holds wealth in two assets, money $M_t$ and bonds. Money pays interest rate of zero, while bonds pay nominal interest $i_t$. Household wealth $A$ then evolves as follows:
        \begin{align}
            A_{t+1} = M_t + (1+i_t)(A_t + W_t L_t - P_t C_t - M_t).
        \end{align}
        
        The household chooses consumption $C_t$, $M_t$, and labor $L_t$ and takes prices $P_t$, wages $W_t$, and interest $i_t$ as given.
        
    
    \subsection{Household Behaviour and IS-LM}
        
        Assume CRRA utility for $U(\cdot)$ and $\Gamma(\cdot)$:
        \begin{align}
            U(C_t)
            &= \frac{C_t^{1-\theta}}{1-\theta},
            \quad \theta > 0,\\
            \Gamma\left(\frac{M_t}{P_t}\right)
            &= \frac{(M_t/P_t)^{1-\chi}}{1-\chi},
            \quad \chi > 0.
        \end{align}
        
        Euler equation for consumption with CRRA utility is then
        \begin{align}
            C_t^{-\theta} = \beta (1+r_t) C_{t+1}^{-\theta}
            \label{eqn:euler-consumption}
        \end{align}
        See Appendix \eqref{calc:euler-consumption} for calculation.
        Taking logs, we have that
        \begin{align}
            -\theta \ln C_t &= \ln \beta +  \ln(1 + r_t) - \theta \ln C_{t+1} \\
            \implies
            \ln C_t &= -\frac{1}{\theta}\left(\beta + \ln(1+r_{t+1})\right) + \ln C_{t+1}
        \end{align}
        
        In this example, all output is consumed (since no capital/investment choice or government spending), therefore $Y_t = C_t$ for all periods. Since $r \simeq \ln(1+r)$ for small $r$'s, we take this as equal. Then
        \begin{align}
            \ln Y_t &= a + \ln Y_{t+1} - \frac{r_t}{\theta}
            \label{eqn:NKIS}
        \end{align}
        where $a = -\frac{\ln \beta}{\theta}$. Equation \eqref{eqn:NKIS} is the \textbf{New Keynesian IS (Investment-Savings) Curve}, which implies an inverse relationship between output $Y_t$ and real interest rate $r_t$, since $\frac{\partial Y_t}{\partial r_t} = -\frac{1}{\theta} < 0$. Optimal money holdings is given by the condition
        \begin{align}
            \Gamma'\left(\frac{M_t}{P_t}\right)
            = \frac{i_t}{1+i_t}U'(C_t)
        \end{align}
        
        With CRRA utilities and $C_t = Y_t$, we have
        \begin{align}
            \frac{Y_t}{M_t} = Y_t^{\theta/\chi} \left(\frac{1+i_t}{i_t}\right)^{1/\chi}.
            \label{eqn:optimal-money-holdings}
        \end{align}
        
        See calculations in \eqref{calc:optimal-money-holdings}. This is the \textbf{LM (liquidity-money supply) curve}. Note that nominal interest rate $i_t \simeq r_t + \pi_t$ for small values. Then in the LM curve, $Y_t$ is directly proportional to $r_t$.
    
    % \subsection{Wage Rigidity}
        
    %     Assuming wages are constant $W_t = \bar{W}$. Then aggregate demand shock $Y_t$
    
    \section{Imperfect Competition}
    
    \subsection{Assumptions}
        
        Assume a continuum of monopolistic firms $i \in [0, 1]$ producing unique goods with production functions
        \begin{align}
            Y_{i} = L_{i}
        \end{align}
        
        Representative household has utility
        \begin{align}
            U &= C - \frac{1}{\gamma} L^{\gamma}, \quad \gamma > 1,
            \\
            C &= \left[
                \int_{i=0}^1 C_i^{(\eta - 1)/\eta} di
            \right]^{\eta / (\eta - 1)},
            \quad \eta > 1.
            \label{eqn:impcomp-consumption-utility}
        \end{align}
        Note that $C$ is not the total consumption, something like a continuous form of CES utility, where $\eta \to -\infty$ implies \textbf{perfect substitutability} of the goods (therefore perfect competition among ``monopolies"). As a simple model of monetary policy, let aggregate demand equal to money supply,
        \begin{align}
            Y = C = \frac{M}{P}.
        \end{align}
        The important feature is that aggregate demand is inversely proportional to prices.
        
    \subsection{Households}
        
        Let $S$ be the household's total spending budget. Household's Lagrangian is then
        \begin{align}
            \L = \left(
                    \int_{i=0}^1 C_i^{(\eta - 1)/\eta} di
                \right)^{\eta / (\eta - 1)}
            - \frac{1}{\gamma} L^{\gamma}
            +\lambda \left(
                S - \int_{i=0}^1 P_i C_i di
            \right).
        \end{align}
        
        Optimizing consumption, we have the FOC
        \begin{align}
            \frac{\partial \L}{\partial C_i} = 0
            &\implies 
            \frac{\eta}{\eta - 1} \left(
                    \int_{j=0}^1 C_j^{(\eta - 1)/\eta} dj
                \right)^{1 / (\eta - 1)} C_i^{\frac{- 1}{\eta}} \frac{\eta - 1}{\eta}
            = \lambda P_i
            \\
            &\implies
            \left(
                    \int_{j=0}^1 C_j^{(\eta - 1)/\eta} dj
                \right)^{1 / (\eta - 1)} C_i^{\frac{- 1}{\eta}}
            = \lambda P_i
            \\
            &\implies
            C_i = A P_i^{-\eta}
            \label{eqn:impcom-consumption}
        \end{align}
        for some $A$ which is constant across firms $i$. Plugging \eqref{eqn:impcom-consumption} into the budget constraint,
        we have
        \begin{align}
            S &= \int_{j=0}^1 P_j C_j dj
            = \int_{j=0}^1 P_j (A  P_j^{-\eta}) dj
            = A \int_{j=0}^1 P_j^{1-\eta} dj
            \\ \implies
            A &= \frac{S}{\int_{j=0}^1 P_j^{(1-\eta)} dj}.
            \label{eqn:A}
        \end{align}
        
        Then we have that
        \begin{align}
            C
            = \left[
                    \int_{i=0}^1 C_i^{(\eta - 1)/\eta} di
                \right]^{\eta / (\eta - 1)}
            % = \left[
            %     \int_{i=0}^1 (A P_i^{-\eta})^{(\eta - 1)/\eta} di
            %     \right]^{\eta / (\eta - 1)}
            = \frac{S}{P}, \quad P \equiv \left[
                \int_{i=0}^1
                P_i^{(1-\eta)}
                di
                \right]^{\frac{1}{1 - \eta}},
                \label{eqn:price-index}
        \end{align}
        where $P$ is the price index corresponding to consumption utility. See Appendix \eqref{calc:price-index} for full derivations. Then we have by \eqref{eqn:impcom-consumption}, \eqref{eqn:A}, and \eqref{eqn:price-index} that
        \begin{align}
            C_i = A P_i^{-\eta}
            = \frac{S}{P^{1-\eta}} P_i^{-\eta}
            = \frac{S}{P} \left(\frac{P_i}{P}\right)^{-\eta}
            = C \left(\frac{P_i}{P}\right)^{-\eta}.
            \label{eqn:consumption-demand}
        \end{align}
    
    \subsection{Firms}
    
        We have that demand for goods $Y_i = C_i = L_i$. Firm $i$ maximizes real profits by choosing prices (monopolistic price setting):
        \begin{align}
            R_i &= \frac{P_i Y_i}{P} - \frac{W L_i}{P}
            = \frac{P_i Y_i}{P} - \frac{W Y_i}{P}
            \\
            &= \frac{P_i Y \left(\frac{P_i}{P}\right)^{-\eta}}{P} - \frac{W Y \left(\frac{P_i}{P}\right)^{-\eta}}{P}
            \\
            &= Y \left(\frac{{P_i}}{P}\right)^{1-\eta} - \frac{WY}{P} \left(\frac{{P_i}}{P}\right)^{-\eta}
        \end{align}
        
        FOC with respect to $\frac{P_i}{P}$:
        \begin{align}
            \frac{\partial R_i}{\partial (P_i / P)} = 0
            &\implies 
            (\eta-1) Y \left(\frac{{P_i}}{P}\right)^{-\eta} 
            = \eta \frac{WY}{P} \left(\frac{{P_i}}{P}\right)^{-\eta-1}
            \\
            &\implies
            \frac{P_i}{P} = \frac{W}{P} \frac{\eta}{\eta-1}.
            \label{eqn:markup-price}
        \end{align}
        
        In perfect competition the firm would be a price taker maximizing profit by choosing $Y_i$, which implies $\frac{P_i}{P} = \frac{W}{P}$, the marginal cost. With imperfect competition, $\frac{\eta}{\eta - 1}$ becomes the markup factor over the perfect competition price. Notice that if $\eta \to -\infty$ then $\frac{P_i}{P} \to \frac{W}{P}$.
        
    
    
    \section{Dynamic Stochastic General Equilibrium Models (DSGE)}
        
    \subsection{Households}
        Representative household maximizes utility
        \begin{align}
            \mathcal{U} &= \sum_{t=0}^{\infty} \beta^{t} [U(C_t) - V(L_t)]
            \quad \beta \in (0, 1),
            \\
            s.t. \quad &
            \sum_{t=0}^{\infty} \frac{P_t C_t}{\prod_{s=0}^{t}(1+r_{s})} \le
            \sum_{t=0}^{\infty} \frac{W_t L_t}{\prod_{s=0}^{t}(1+r_{s})}
        \end{align}
        
        where
        \begin{align}
            U(C_t) &= \frac{C_t^{1-\theta}}{1-\theta},
            \quad
            V(L_t) = \frac{B}{\gamma} L_t^\gamma, 
            \quad 
            \theta, B, \gamma > 0
            \\
            \implies
            U'(C_t) &= C_t^{-\theta},
            \quad
            V'(L_t) = B L_t^{\gamma - 1}.
        \end{align}
        
        Forming Lagrangian, we have
        \begin{align}
            \L &= \sum_{t=0}^{\infty} \beta^{t} [U(C_t) - V(L_t)] + \lambda 
            \left(
                \sum_{t=0}^{\infty} \frac{W_t L_t}{\prod_{s=0}^{t}(1+r_{s})}
                - 
                \sum_{t=0}^{\infty} \frac{C_t}{\prod_{s=0}^{t}(1+r_{s})}
            \right).
        \end{align}
        
        We have the following from the FOCs:
        \begin{align}
            \frac{\partial \L}{\partial L_t} = 0
            &\implies
            \beta^t V'(L_t) = \frac{\lambda W_t}{\prod_{s=0}^{t}(1+r_{s})},
            \\
            \frac{\partial \L}{\partial C_t} = 0
            &\implies
            \beta^t U'(C_t) = \frac{\lambda P_t}{\prod_{s=0}^{t}(1+r_{s})},
            \\
            &\implies
            \frac{W_t}{P_t} = \frac{V'(L_t)}{U'(C_t)}
            = \frac{BL_t^{\gamma - 1}}{C_t^{-\theta}}
            % = BL_t^{\gamma - 1}C_t^{\theta}
            = B Y_t ^{\theta + \gamma - 1}
            \label{eqn:eqm-wage}
        \end{align}
        
        since we assume that $Y_t = F(L_t) = L_t = C_t$. % Our assumptions about consumption are same as in \eqref{eqn:NKIS}, thus again we have
        % \begin{align}
        %     \ln Y_t &= a + \ln Y_{t+1} - \frac{r_t}{\theta}.
        % \end{align}
        
    
    \subsection{Firms}
    
        Firm producing good $i$ has production function
        \begin{align}
            Y_{it} = L_{it}
        \end{align}
        and by \eqref{eqn:consumption-demand} is met with demand
        \begin{align}
            Y_{it} = Y_t \left(\frac{P_{it}}{Pt}\right)^{-\eta}.
            \label{eqn:markup-demand}
        \end{align}
        
        Firm $i$ sets price $p_i$ in period 0. The firm's real profit in period $t$ is
        \begin{align}
            R_{it}(P_i)
            &= \frac{P_i}{P}Y_{it} - \frac{W}{P} L_{it}
            = \frac{P_i}{P}Y_{it} - \frac{W}{P} Y_{it}
            \\
            &=
            Y_t \left(\frac{{P_i}}{P_t}\right)^{1-\eta} - \frac{W_tY_t}{P_t} \left(\frac{{P_i}}{P_t}\right)^{-\eta}
        \end{align}
        
    \subsection{Sticky Prices}
    
        Let $q_t$ be the probability that the price the firm sets in period $0$ stays the same in period $t$. Define the stochastic discount factor (SDF):
        \begin{align}
            \lambda_t = \beta^t \frac{U'(C_t)}{U'(C_0)}.
        \end{align}
        % (?) which comes from Lagrangian
        % \begin{align}
        %     \max_{\{C_t\}_{t=0}^\infty, \{L_t\}_{t=0}^\infty} &
        %     \sum_{t=0}^{\infty} \beta^t \left[ U(C_t) - V(L_t) \right]
        %     \quad s.t. \quad
        %     Y_t \ge C_t, \quad \forall t \in \{0, 1, 2, ...\}
        %     \\
        %     \L
        %     &= \sum_{t=0}^{\infty} \beta^t \left[ U(C_t) - V(L_t) \right] - \sum_{t=0}^\infty \lambda_t (Y_t - C_t)
        %     \\
        %     \frac{\partial \L}{\partial C_t} &= 0
        %     \implies \beta^t U'(C_t) = \lambda_t
        %     \implies \beta^t \frac{U'(C_t)}{U'(C_0)} = \frac{\lambda_t}{\lambda_0}
        % \end{align}
        
        Firm's problem is to choose $p_i$ at time $t=0$ to maximize $A$:
        \begin{align}
            \max_{P_i} A = \E{\sum_{t=0}^\infty q_t \lambda_t R_{it}}
            &= \E{\sum_{t=0}^\infty q_t \lambda_t
            \left(
                Y_t \left(\frac{{P_i}}{P_t}\right)^{1-\eta} - Y_t\frac{W_t}{P_t} \left(\frac{{P_i}}{P_t}\right)^{-\eta}
                \right)}
            \\
            &= \E{\sum_{t=0}^\infty q_t \lambda_t
            Y_t P_t^{\eta - 1}\left(
                P_i^{1-\eta} - \frac{W_t}{P_t} P_i^{-\eta}
                \right)}
            \\
            &= \E{\sum_{t=0}^\infty q_t \lambda_t
            Y_t P_t^{\eta - 1} F(p_i, p_t^*)}
        \end{align}
        where $P_t^*$ is the price that optimizes profits in period $t$.  For example, by \eqref{eqn:markup-price} $P_{it}^* = P_{t}^* = W_t\frac{\eta}{\eta-1}$ for all $i$. $F(p_t, p_t^*)$ is a function of log prices $p_i = \ln{P_i}$ and $p_t^* = \ln{P_t^*}$, where $p_i = p_t^*$ maximizes period-$t$ profits. This implies
        \begin{align}
            \frac{\partial F(p_t^*, p_t^*)}{\partial p_i} = 0,
            \quad
            \frac{\partial^2 F(p_t^*, p_t^*)}{\partial p_i^2} < 0,
        \end{align}
        
        by existence of maximum, assuming $F(p_i, p_t^*)$ is differentiable everywhere. Assume $F(p_i, p_i^*)$ can be approximated around $p_i = p_i^*$ by the second order Taylor approximation:
        \begin{align}
            F(p_i, p_i^*)
            &\simeq F(p_t^*, p_t^*) +  \frac{\partial F(p_t^*, p_t^*)}{\partial p_i} (p_i - p_t^*) + \frac{\partial^2 F(p_t^*, p_t^*)}{\partial p_i^2} (p_i - p_t^*)^2
            \\
            &= F(p_t^*, p_t^*) - K (p_i - p_t^*)^2, \quad K > 0
            \\
            \implies
            F(p_t^*, p_t^*) - F(p_i, p_i^*) &= K(p_i - p_t^*)^2.
        \end{align}
        
        Assuming fluctuations in $Y_t P_t^{\eta - 1}$ is negligible across periods compared to $q_t$ and $F(p_i, p_t^*)$, we can find the optimal $p_i = \ln P_i$ by minimzing the ``distance" between $F(p_i, p_t^*)$ and $F(p_t^*, p_t^*)$:
        \begin{align}
            p_i^*
            &= \arg\min_{p_i} \sum_{t=0}^\infty q_t \beta^t \E{F(p_i, p_t^*) - F(p^*, p^*)}
            \\
            % &= \arg\min_{p_i} \sum_{t=0}^\infty q_t \beta^t \E{(p_i - p_t^*)^2}
            % \\
            \implies p_i^*
            &= \sum_{t=0}^\infty \tilde\omega_t \E{p_t^*}, \quad
            \tilde\omega_t = \frac{\beta^t q_t}{\sum_{s=0}^\infty \beta^s q_s}
            \label{eqn:p-star}
        \end{align}
        
        See appendix \eqref{calc:p-star} for calculations. Then combining equilibrium log-wage $w_t$ from \eqref{eqn:eqm-wage} and each firm's profit-maximizing log-price in each period $p_t^*$ from \eqref{eqn:markup-price}, we have
        \begin{align}
            w_t &= p_t + \ln{B} + (\theta + \gamma - 1) y_t,
            \\
            p_t^* &= \ln{\frac{\eta}{\eta - 1}} + w_t
            = \phi m_t + (1-\phi) p_t + c
            \label{eqn:optimal-price-t}
        \end{align}
        
        where $m_t = y_t + p_t$ is log-nominal GDP and
        \begin{align}
            \phi &= \theta + \gamma - 1 > 0, \quad
            c = \ln{\frac{\eta}{\eta - 1}} + \ln{B} = 0
        \end{align}
        
        See appendix \eqref{calc:optimal-price-t} for full derivation. Then firm $i$'s optimal price at time $t$ is
        \begin{align}
            p_{it}^*
            = \sum_{s=0}^\infty \tilde\omega_{t+s} \E{p_{t+s}^*}
            &= \sum_{s=0}^\infty \tilde\omega_{t+s} \E{\phi m_{t+s} + (1-\phi) p_{t+s}}.
        \end{align}
        
    \subsection{Calvo Model and New Keynesian Phillips Curve}
    
        Assume in every period, a random fraction $\alpha \in (0, 1]$ of firms can change prices.\footnote{Probability that price stays the same after $j$ periods are then $q_j = (1-\alpha)^j$.}
        Average prices $p_t$ and inflation $\pi_t$ in period $t$ is then
        \begin{align}
            p_t &= \alpha x_t + (1 - \alpha) p_{t-1}
            \\
            \implies
            p_t - p_{t-1}
            &= \pi_t = \alpha (x_t - p_{t-1})
            \implies
            \frac{\pi_t}{\alpha} = x_t - p_{t-1}
        \end{align}
        where $p_{t-1}$ is average old price and $z_t$ is new profit-mazimizing prices\footnote{$y = 1 + z + z^2 + z^3 + ... = 1 + z(1 + z + z^2 + z^3 + ...) = 1 + zy \implies y = \frac{1}{1-z}$. Then letting $z = \beta(1-\alpha)$, we have $\left(\sum_{k=0}^\infty\beta^k(1-\alpha)^k\right)^{-1} = 1-\beta(1-\alpha)$.}
        
        \begin{align}
            x_t
            &= \sum_{j=0}^\infty \tilde\omega_{j} \E{p_{t+j}^*}
            ,\quad
            \tilde\omega_j
            = \frac{\beta^j (1-\alpha)^j}{\sum_{k=0}^\infty \beta^k (1-\alpha)^k}
            = [1 - \beta(1-\alpha)] \beta^j (1-\alpha)^j q_j
            \\
            x_t
            &= [1 - \beta(1-\alpha)] \sum_{j=0}^\infty \beta^j (1-\alpha)^j \E{p_{t+j}^*}
            \\
            &= [1 - \beta(1-\alpha)] \E{p_{t}^*}
            + \beta (1-\alpha) [1 - \beta(1-\alpha)] \sum_{j=0}^\infty \beta^j (1-\alpha)^j \E{p_{(t+1)+j}^*}
            \\
            &= [1 - \beta(1-\alpha)] p_{t}^*
            + \beta (1-\alpha) \E{x_{t+1}}
            \\
            \implies
            x_t - p_t
            &= (x_t - p_{t-1}) - (p_t - p_{t-1})
            = \frac{\pi_t}{\alpha} - \pi_t
            = \pi_t \frac{1 - \alpha}{\alpha}
            \\
            &= [1 - \beta(1-\alpha)] (p_{t}^* - p_t)
            + \beta (1-\alpha) \E{x_{t+1} - p_t}
            \\
            &= [1 - \beta(1-\alpha)] (\phi y_t) 
            + \beta (1-\alpha) \E{\frac{\pi_{t+1}}{\alpha}}
            \? % why is \phi y_t = p_t^* - p_t ?
            \\
            \implies
            \pi_t
            &=
            \frac{\alpha}{1-\alpha}
            [1 - \beta(1-\alpha)] \phi y_t
            + \beta \E{\pi_{t+1}}
            \\
            &= \kappa y_t + \beta \Et{\pi_{t+1}},
            \quad
            \kappa = \frac{\alpha}{1-\alpha}
            [1 - \beta(1-\alpha)] \phi
            \label{eqn:nk-phillips-curve}
        \end{align}
        Higher output raises inflation as well as higher future expectations of inflation.
        
    \section{Monetary Policy}
    
    \subsection{Inflation, money growth, interest rates}
        
        Model for LM curve in \eqref{eqn:optimal-money-holdings} implies real money demand is decreasing in nominal interest $i$ and increasing in real income $Y$. Write demand for real balances
        \begin{align}
            \frac{M}{P} &= L(i, Y),
            \quad \frac{\partial L}{\partial i} < 0,
            \quad \frac{\partial L}{\partial Y} > 0
            \\
            \implies
            P &= \frac{M}{L(i, Y)},
        \end{align}
        where $M$ is money stock and $P$ is price level. Real interest is defined as nominal interest minus expected inflation, which gives us the \textbf{Fischer identity}:
        \begin{align}
            r \equiv i - \pi^e
            &\implies
            i \equiv r + \pi^e
            \\
            &\implies
            P = \frac{M}{L(r + \pi^e, Y)},
        \end{align}
        
        where $r$ is the real interest rate and $\pi^e$ is expected inflation. Assume $\pi^e = \frac{\dot P}{P} = \frac{d \ln P}{d t}$ is actual inflation, and $r$ and $Y$ are constant at $\bar{r}$ and $\bar{Y}$.
        \begin{align}
            \ln P = \ln M - \ln L(\bar r + \pi^e, \bar Y)
            &\implies
            \frac{d \ln{P}}{d t} = \frac{\dot{P}}{P} = \pi^e = \frac{\dot M}{M} - \frac{\dot L}{L}
        \end{align}
        
        Then if there a permanent increase in growth rate of nominal money supply $\frac{d \ln M}{d t} = \frac{\dot{M}}{M}$ at $t_0$, we have
        \begin{align}
            \frac{\dot M}{M} \uparrow
            & \implies
            \frac{\dot P}{P} = \pi^e \uparrow
            \label{eqn:fischer-effect}
            \implies
            i = \bar r + \pi^e \uparrow
            \implies
            \frac{\dot L}{L} \downarrow,
            \quad \because
            \frac{\partial L}{\partial i} < 0
        \end{align}
        
        The first implication is the \textbf{Fischer effect}, which shows that inflation affects nominal rate one-for-one $M \uparrow \implies \pi^e \uparrow \implies i \uparrow$ by the Fischer identity and the assumption that inflation does not affect real interest rate.
        
    \subsection{Expectation Theory of Term structure}
        
        Let $i_t^n$ be the continuously compounded $n$-period (zero-coupon) interest rates at period $t$. If future interests rates are certain, in equilibrium (no arbitrage) we must have that the one-period compounded returns are the same:
        \begin{align}
            e^{n \cdot i_t^n }
            &= e^{i_t^1 } \times e^{i_{t + 1}^1 } \times ... \times e^{i_{t + n - 1}^1 }
            = e^{i_t^1 + i_{t + 1}^1 + ... + i_{t + n - 1}^1 }
            \\
            \implies
            n i_t^n &= i^1_{t} + i^1_{t+1} + ... + i^1_{t+n-1}.
        \end{align}
        
        With uncertainty about future one term interest rates, in equilibrium we have
        \begin{align}
            i^{n}_t
            = \frac{1}{n} \sum_{j=0}^{n-1} \Et{i^1_{t+j}}
            = \frac{i^1_t + \Et{i^1_{t+1}} + ... + \Et{i^1_{t+n-1}}}{n} + \theta_{nt}
        \end{align}
        
        where $i_t^n$ is the continuously compounded interest rate for $n$ periods starting from period $t$,  and $\theta_{nt}$ is the \textbf{term premium} for uncertainty of future one-period interest rates $i^1_{t+j}$.
        
    \subsection{Stabilization Policy}
    
        Assumed welfare function
        \begin{align}
            W_t = - c(y^*_t - y_t)^2  - f(\pi_t - \pi^*),
            \quad c > 0, \quad f(\cdot) > 0
        \end{align}
        
        where $y_t, \pi_t$ are \textbf{actual output} and inflation and $y^*_t, \pi^*$ are \textbf{target (Walrasian\footnote{A Walrasian model is a model of competitive markets without externalities, asymmetric information, missing markets, or other imperfections.}) output} and inflation. Policy maker's goal is to minimize deviations from target. \\
        
        
        Consider the New Keynesian Philips curve (NKPC) from  \eqref{eqn:nk-phillips-curve} and the New Keynesian Investment-Savings curve (NKIS) from \eqref{eqn:NKIS}
        \begin{align}
            \pi_t
            &= \beta \Et{\pi_{t+1}} + \kappa (y_t - y^n_t),
            \quad
            \kappa  > 0, \quad \beta \in (0, 1)
            \label{eqn:NKPC-2}
            \\
            y_t
            &= \Et{y_{t+1}} - \frac{r_t}{\theta}
            + u_t^{IS}
            \\
            &= \Et{y_{t+1}} - \frac{1}{\theta}(i_t - \Et{\pi_{t+1}} -\rho) + u_t^{IS},
            \quad e^{-\rho} = \beta, \theta > 0
            \label{eqn:NKIS-2}
        \end{align}
        
        where $y^n_t$ is the flexible-price (or ``natural") level of output, and
        \begin{align}
            y^n_t &= \rho_Y y^n_{t-1} + \varepsilon^Y_t,
            \quad \rho_Y \in (0, 1),
            \\
            u_t^{IS} &= \rho_{IS} u_{t-1} + \varepsilon^{IS}_t,
            \quad \rho_{IS} \in (0, 1).
        \end{align}
        
        Assume central bank wants $y_t = y^n_t$ and $\pi_t = 0$ for all periods $t$. This requires
        \begin{align}
            \pi_t
            &= \Et{\pi_{t+1}} = 0,
            \\
            y_t
            &= y_t^n, \quad \E{y_{t+1}} = \E{y_{t+1}^n},
        \end{align}
        which implies that the NKPC \eqref{eqn:NKPC-2} holds. Then plugging in the conditions above, the optimal  nominal interest rate policy is
        \begin{align}
            i_t
            &= \rho + \theta\left(\Et{y_{t+1}^n} - y_t^n + u_t^{IS}\right)
            \\
            &= r_t^n + \pi_t = r_t^n.
        \end{align}
        
        Sunspot equilibria are possible with this policy. For example, suppose inflation and output jump up at time $t$, and are expected to return to normal in $t+1$ (though still out of original equilibrium).
        
        \begin{align}
            \pi_t
            \uparrow
            \pi_t',
            &\quad
            \Et{\pi_{t+1}} \uparrow
            \Et{\pi_{t+1}'},
            \quad
            \pi_t' > \Et{\pi_{t+1}'} > 0,
            \\
            y_t \uparrow y_t',
            &\quad
            \Et{y_{t+1}} \uparrow \Et{y_{t+1}'},
            \qquad
            \Et{y_t'} > \Et{y_t^n}, \quad
            \Et{y_{t+1}'} > \Et{y_{t+1}^n}
        \end{align}
        
        ++
        \begin{align}
            i_t = r_t^n = r_t + \Et{\pi_{t+1}},
            \quad
            \Et{\pi_{t+1}} \uparrow
            &\implies
            r_t \downarrow
            \\
            &\implies
            y_t = y_t^n
            = \Et{y_{t+1}} - \frac{r_t}{\theta} + u_t^{IS}
            \\
            &\implies
            \Et{y_{t+1}} \downarrow
            \\
            \pi_t = \beta \Et{\pi_{t+1}} + \kappa (y_t - y_t^n) = 0
            &\implies
            y_t = y_t^n - \frac{\beta}{\kappa}\Et{\pi_{t+1}} > y_t^n
        \end{align}
        
    \newpage
    \appendix
    \section{Calculation for (\ref{eqn:euler-consumption})}\label{calc:euler-consumption}
        \begin{align}
            \L(C_t, M_t, \lambda)
            &= \sum_{t=0}^\infty \beta^t
            \left[
                \frac{C_t^{1-\theta}}{1-\theta}
                + \frac{(M_t/P_t)^{1-\chi}}{1-\chi}
                - V(L_t)
            \right] \\
            &\quad + \lambda \left(
                \sum_{t=0}^\infty \frac{W_t L_t}{\prod_{s=0}^{t}(1+r_s)} - \frac{C_t}{\prod_{s=0}^{t}(1+r_s)}
            \right).
            \\
            \frac{\partial\L}{\partial C_t} = 0
            &\implies \beta^t C_t^{-\theta}
            = \frac{\lambda}{\prod_{s=0}^{t}(1+r_s)}
            \\
            &\implies
            \frac{\beta^t C_t^{-\theta}}{\beta^{t+1} C_{t+1}^{-\theta}}
            =  \frac{\lambda \prod_{s=0}^{t+1}(1+r_s)}{\lambda \prod_{s=0}^{t}(1+r_s)}
            \\
            &\implies
            C_t^{-\theta}
            =  \beta (1+r_{t+1}) C_{t+1}^{-\theta}
        \end{align}
        
        Change time indexing convention for $r_{t+1}$ to one period prior. $r_{t}$ is the interest to be paid out at $t+1$. Then the Euler equation is
        \begin{align}
            C_t^{-\theta}
            =  \beta (1+r_t) C_{t+1}^{-\theta}.
        \end{align}
    
    
    \section{Calculation for (\ref{eqn:optimal-money-holdings})}\label{calc:optimal-money-holdings}
    
    \begin{align}
        A_{t+1}
        &= M_t + (1+i_t)(A_t + W_t L_t - P_t C_t - M_t)
        \\
        &= [M_t + \Delta m] + (1+i_t)(A_t + W_t L_t - P_t C_t - [M_t + \Delta m]) - i_t \Delta m
        \\
        &= [M_t + \Delta m]
        + (1+i_t)\left(
          A_t + W_t L_t - P_t C_t - [M_t + \Delta m] - \frac{i_t}{1+i_t} \Delta m \right)
         \\
         &= [M_t + \Delta m]
        + (1+i_t)\left(
          A_t + W_t L_t - P_t \left[C_t + \frac{i_t \Delta m}{1+i_t} \right] - [M_t + \Delta m] \right)
    \end{align}
    
    Thus a change of $\Delta m$ in $M_t$ results in a change of $\frac{i_t}{1+i_t} \Delta m$ in $C_t$ within the same budget.
    Then in one period $t$,
    \begin{align}
        d \mathcal{U} = 0 = U'(C_t) + \Gamma'(M_t / Y_t)
    \end{align}
    
    \section{Calculation for (\ref{eqn:price-index})}\label{calc:price-index}
    
    From \eqref{eqn:impcom-consumption} we have $C_i = A P_i^{-\eta}$, and from \eqref{eqn:A} we have $A = \frac{S}{\int_{j=0}^1 P_j^{(1-\eta)} dj}$. Then with the definition of aggregate consumption $C$ from \eqref{eqn:impcomp-consumption-utility}, we have
    \begin{align}
            C = \left[
                    \int_{i=0}^1 C_i^{(\eta - 1)/\eta} di
                \right]^{\eta / (\eta - 1)}
            &= \left[
                \int_{i=0}^1 (A P_i^{-\eta})^{(\eta - 1)/\eta} di
                \right]^{\eta / (\eta - 1)}
            \\
            &= \left[
                \int_{i=0}^1 A^{(\eta - 1)/\eta} P_i^{(1-\eta)} di
                \right]^{\eta / (\eta - 1)}
            \\
            &= \left[
                \int_{i=0}^1
                \left(\frac{S}{\int_{j=0}^1 P_j^{(1-\eta)} dj}\right)^{(\eta - 1)/\eta}
                P_i^{(1-\eta)}
                di
                \right]^{\eta / (\eta - 1)}
            \\
            &= \frac{S}{\int_{j=0}^1 P_j^{(1-\eta)} dj}
                \left[
                \int_{i=0}^1
                P_i^{(1-\eta)}
                di
                \right]^{\eta / (\eta - 1)}
            \\
            &= \frac{S}{
                \left[
                \int_{i=0}^1
                P_i^{(1-\eta)}
                di
                \right]^{\frac{1}{1 - \eta}}
                }
            \\
            &= \frac{S}{P}, \quad P \equiv \left[
                \int_{i=0}^1
                P_i^{(1-\eta)}
                di
                \right]^{\frac{1}{1 - \eta}}.
        \end{align}
        
        
    \section{Calculation for (\ref{eqn:p-star})}\label{calc:p-star}
        Expectation of the square of a random $x$ is as follows
        \begin{align}
            \E{x^2}
            &= \E{([x - \mu_x] + \mu_x)^2}
            \\
            &= \E{\mu_x^2 + (x - \mu_x)^2 + 2\mu_x(x - \mu_x)}
            \\
            &= \mu_x^2 + \E{(x - \mu_x)^2} + 2\mu_x\E{x - \mu_x}
            \\
            &= \E{x}^2 + \var{x}.
        \end{align}
    
        Then,
        \begin{align}
            p_i^*
            = \arg\min_{p_i} \sum_{t=0}^\infty q_t \beta^t \E{F(p_i, p_t^*) - F(p^*, p^*)}
            &=\sum_{t=0}^\infty q_t \beta^t \E{(p_i - p_t^*)^2}
            \\
            &= \sum_{t=0}^\infty q_t \beta^t \left[(p_i - \E{p_t^*})^2 + \var{p_t^*}\right].
        \end{align}
        
        Then finding optimal $p_i$ with the FOC:
        \begin{align}
            \frac{\partial D}{\partial p_i}
            = \sum_{t=0}^\infty 2 \beta^t q_t (p_i - \E{p_t^*})
            = 0
            &\implies
            2p_i \sum_{s=0}^\infty \beta^s q_s = 2\sum_{t=0}^\infty q_t \beta^t \E{p_t^*}
            \\
            &\implies
            p_i = \frac{\sum_{t=0}^\infty q_t \beta^t \E{p_t^*}}{\sum_{s=0}^\infty \beta^s q_s}
            = \sum_{t=0}^\infty \frac{\beta^t q_t}{\sum_{s=0}^\infty \beta^s q_s} \E{p_t^*}
            \\
            &\implies
            p_i^* = \sum_{t=0}^\infty \tilde\omega_t \E{p_t^*}, \quad \tilde\omega_t = \frac{\beta^t q_t}{\sum_{s=0}^\infty \beta^s q_s}
        \end{align}
        
        \section{Calculation for (\ref{eqn:optimal-price-t})}
        \label{calc:optimal-price-t}
        
        From \eqref{eqn:eqm-wage} we have
        \begin{align}
            w_t &= p_t + \ln{B} + (\theta + \gamma - 1) y_t,
        \end{align}
        
        and from \eqref{eqn:markup-price} we have each firm's profit-maximizing log-price in each period,
        \begin{align}
            \max_{p_{it}}\ln[R_{it}(P_{it})]
            = p_t^* &= \ln{\frac{\eta}{\eta - 1}} + w_t
            \\
            &= \ln{\frac{\eta}{\eta - 1}}
            + p_t + \ln{B} + (\theta + \gamma - 1) y_t
            \\
            &= p_t + c + \phi y_t
            \\
            &= p_t + c + \phi y_t + \phi p_t - \phi p_t
            \\
            &= \phi (y_t + p_t) + (1-\phi) p_t + c
            \\
            &= \phi m_t + (1-\phi) p_t, \quad m_t = y_t + p_t, \quad c = 0
        \end{align}
        where $\phi = \theta + \gamma - 1$ and $c = \ln{\frac{\eta}{\eta - 1}} + \ln{B}$. $m_t$ is log nominal GDP $y_t + p_t$.
        
\end{document}